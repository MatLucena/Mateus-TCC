\chapter{Conclusão} \label{cap7}

\section{Considerações Finais}
Com este projeto conseguimos evidenciar a importância da identificação de sistemas por métodos caixa-preta para o controle da altura de uma bola dentro de um túnel de vento. Exploramos a possibilidade de usar um modelo caixa branca para modelar o sistema mas percebemos que o seu desempenho estava muito aquém do sistema real. Somente a aplicação de identificação caixa-preta, no entanto, não é suficiente para resolver o problema da modelagem, é preciso validar o modelo escolhido e determinar se o método de identificação é adequado para o tipo de sistema estudado.


O desenvolvimento deste projeto apresentou diversos desafios na construção adequada do sistema estudado. O diâmetro da bola de tênis de mesa foi um fator que ao oscilar dentro do tubo, foi capaz de mudar drasticamente a leitura do sensor. Levando em conta que a oscilação foi da ordem de 2 mm, ainda foi suficiente para interferir no experimento. O motor RC usado, tanto quanto outros motores, não foi capaz de manter uma velocidade constante, o que influenciou nos testes.


A manufatura do tubo do túnel de vento também precisa de certas considerações a serem tomadas como o escape de ar necessário para que a bola possa ser controlada. Foi utilizada uma furadeira de bancada para alinhar os furos verticais e tentar garantir que a distância entre os mesmos fosse igual.


Constatamos que o método de identificação e controle para este tipo de sistema não é adequado para garantir os requisitos de tempo de assentamento e máximo sobrevalor. O sistema estudado é não linear e apresenta uma dinâmica difícil de identificar.




\section{Trabalhos Futuros}
Levando em conta os conhecimentos adquiridos durante este trabalho percebemos ser necessário o projeto de um novo túnel de vento, com um duto maior que o atual, com uma bola maior com a mesma relação de diâmetro entre a bola atual e o tubo, para diminuir o erro de medição do sensor. O tubo maior também influencia na medida do sensor diminuindo a chance de haver reflexão nas paredes e nos buracos de escape de ar. 




% Fim Capítulo
