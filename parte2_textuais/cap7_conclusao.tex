\chapter{Conclusão} \label{cap7}

\section{Considerações Finais}
Com este projeto conseguimos evidenciar a importância da identificação de sistemas por métodos caixa-preta para o controle da altura de uma bola dentro de um túnel de vento. Exploramos a possibilidade de usar um modelo caixa branca para modelar o sistema mas percebemos que o seu desempenho estava muito aquém do sistema real. Somente a aplicação de identificação caixa-preta, no entanto, não é suficiente para resolver o problema da modelagem, é preciso validar o modelo escolhido e determinar se o método de identificação é adequado para o tipo de sistema estudado.


O desenvolvimento deste projeto apresentou diversos desafios na construção adequada do sistema estudado. O diâmetro da bola de tênis de mesa foi um fator que ao oscilar dentro do tubo, foi capaz de mudar drasticamente a leitura do sensor. Levando em conta que a oscilação foi da ordem de 2 mm, ainda foi suficiente para interferir no experimento. O motor RC usado, tanto quanto outros motores, não foi capaz de manter uma velocidade constante, o que influenciou nos testes.


A manufatura do tubo do túnel de vento também precisa de certas considerações a serem tomadas como o escape de ar necessário para que a bola possa ser controlada. Foi utilizada uma furadeira de bancada para alinhar os furos verticais e tentar garantir que a distância entre os mesmos fosse igual.


Constatamos que o método de identificação e controle para este tipo de sistema não é adequado para garantir os requisitos de tempo de assentamento e máximo sobrevalor. O sistema estudado é não linear e apresenta uma dinâmica difícil de identificar.




\section{Trabalhos Futuros}

Levando em conta os conhecimentos adquiridos durante este trabalho percebemos ser necessário o projeto de um novo túnel de vento. O diâmetro da bola de tênis de mesa é muito pequeno para o sensor, qualquer oscilação lateral da bola é suficiente para que o sensor meça errado. Uma forma de resolver isso é utilizando um tubo menor. Entretanto, isso traz um novo problema, o cone de medida do sensor, apesar de estreito, precisa ser alinhado com o tubo, e quanto menor o tubo mais sensível é o alinhamento. A forma mais sensata de resolver o problema da oscilação da bola é utilizar um duto de diâmetro maior e uma bola maior. Isso faz com que o infravermelho tenha uma superfície de reflexão maior e que a sensibilidade de alinhamento diminua.


É interessante implementar um controlador para a velocidade do motor, que garanta que o vento esteja fluindo exatamente na velocidade desejada. Não pudemos determinar o quanto a falta de um controle de velocidade influenciou nos experimentos, ou mesmo o quão correta foi a velocidade do motor. Mas. ao enviarmos um PWM ao motor, podemos ver visualmente, após o sistema estabilizar naturalmente, que a bola oscila levemente em torno de um ponto. Esta oscilação aumenta de acordo com o aumento da velocidade, e, portanto, atribuímos à falta de controle da velocidade do motor.


As medidas discrepantes do sensor podem ser tratadas com um filtro digital, o que poderia resolver os problemas de leitura do sensor. Identificamos que os modelos são deficientes fora do ponto de operação, isso pode ser tratado por técnicas adaptativas como MPC, ou a partir de redes de modelos locais, como modelos Takagi Sueno.



% Fim Capítulo
