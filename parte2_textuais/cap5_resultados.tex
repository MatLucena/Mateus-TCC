\chapter{Título do Capítulo Aqui} \label{cap5}
RESULTADOS AQUI
\newline
Espaço de Hilbert
É um espaço de produto interno real ou complexo \newline
É um espaço métrico completo a respeito da função de distância do produto interno.
\newline
O produto interno de um par de elementos é igual ao conjugado complexo do produto interno dos elementos trocados.
\begin{equation}
\langle y,x \rangle=\overline{\langle x,y \rangle}
\end{equation}
O produto interno é linear no primeiro argumento para todos os numeros complexos a e b.
\begin{equation}
\langle ax_1+bx_2,y \rangle=a\langle x_1,y \rangle + b \langle x_2,y\rangle
\end{equation}
O produto interno de um elemento com si mesmo é definido positivo
\begin{equation}
\langle x,x \rangle \geqslant 0
\end{equation}

a e b são elementos de $\mathcal{H}$ e $\mathcal{A, B}$ subespaços de $\mathcal{H}$.


Lemma 9.1: Seja $\mathcal{ B, C} \subset \mathcal{H}$

























% Fim Capítulo
