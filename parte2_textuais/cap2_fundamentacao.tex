\chapter{ Fundamentação Teórica} \label{cap2}
Neste capítulo serão apresentados conceitos necessários para o entendimento do trabalho.

\section{Identificação de Sistemas e Estimação de Parâmetros}
A identificação do sistema é o primeiro passo para o seu controle. Nesta seção serão tratados conceitos de identificação de sistemas e estimação de parâmetros fundamentais para o entendimento do trabalho.

\subsection{Visão Geral}
A identificação de sistemas e estimação de parâmetros se tratam de métodos e práticas que permitem construir modelos dinâmicos de um sistema real à partir de experimentos. Muitas vezes um sistema construído que precisa ser controlado não pode ser modelado devido à limitações matemáticas ou imprecisão na interação dos componentes. Nestes casos se utiliza da identificação de sistemas para obter um modelo matemático. A identificação de sistemas se baseia em testar a resposta do sistema à certas entradas e a partir das respostas aproximar o modelo matemático de forma satisfatória. Para identificar sistemas temos métodos determinísticos, que desprezam o ruído presente nos dados, e métodos não paramétricos, que não resultam em um modelo matemático mas em uma representação gráfica da dinâmica do sistema da qual um modelo pode ser extraído.


\subsection{Identificação por Mínimos Quadrados}
O método de mínimos quadrados é um dos mais conhecidos e utilizados em várias áreas da ciência e tecnologia. Ele utiliza sistemas de equações com matrizes geradas a partir de testes com os sistemas reais no seguinte formato:
\begin{equation}
	\hat{\Theta}=[X^TX]^{-1}X^Ty
\end{equation}
A equação 2.1 é o objetivo da identificação por quadrados mínimos, mas para entender ela precisamos primeiro entender de onde ela vem.
\newline 
Um sistema geralmente é descrito por uma função entrada - saída, em um sistema onde não sabemos a função podemos executar uma série de testes para obtermos um conjunto de valores de entrada e saída da seguinte forma:
\begin{equation}
\begin{array}{c}
y_1=f(x_1) \\
y_2=f(x_2) \\
y_3=f(x_3) \\
... \\
y_N=f(x_N)
\end{array}
\end{equation}
Podemos então tratar esse conjunto como vetores da seguinte forma:
\begin{equation}
y=f(x,\Theta)
\end{equation}
Onde $ \Theta $ é um vetor de n parâmetros.



% Fim Capítulo















