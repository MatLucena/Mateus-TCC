\chapter{Identificação do Túnel de Ar} \label{cap4}
Neste capítulo será explicada a identificação do sistema do túnel de ar. Para fazer a identificação caixa-preta de um sistema é necessário fazer um estudo prévio do funcionamento dele, conhecer entradas e saídas, obter um modelo matemático se possível, mesmo sem ter todos os parâmetros.
\section{Escolha de estrutura}
A escolha da estrutura de uma identificação se dá a partir das informações pesquisadas sobre o sistema, quantidade de entradas, quantidade de saídas, perturbações, modelo básico. Escolhemos duas estruturas, uma para cada forma de identificação.
\subsection{Mínimos Quadrados}
A identificação por mínimos quadrados possibilita 
\subsection{Subespaços}
\section{Experimento}
\section{Estimação}
\subsection{Mínimos Quadrados}
\subsection{Subespaços}
\section{Validação}



% Fim Capítulo