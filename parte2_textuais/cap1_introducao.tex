\pagenumbering{arabic}

\chapter{Introdução} %Contextualização, motivação e justificativa.

\section{Levitação a ar}

O túnel de vento é uma instalação utilizada para estudar o desempenho aerodinâmico de objetos. Ele consiste em um duto de tamanho adequado para o objeto testado pelo qual o ar flui ao ser empurrado por uma turbina. Os testes feitos variam de velocidade, de acordo com o que se deseja estudar.


Para que os testes sobre objetos tenham resultados precisos e replicáveis, a velocidade dos ventos sobre eles deve ser controlável e o fluxo de ar que atinge o objeto não deve ter turbulência, apesar de haver testes onde isso é proposital. O túnel é construído com um dispositivo de fluxo laminar que possibilita reduzir o fluxo turbulento gerado pela turbina \cite{mcdade1969}.


Os testes geralmente são feitos em modelos em escala de aviões, carros, naves espaciais e para  estudo da aerodinâmica de objetos na física. Os túneis podem ser verticais ou horizontais, apesar de os túneis horizontais existirem em maior quantidade. Um teste comum é feito em modelos de aviões onde se medem as propriedades das asas como envergadura ao ser submetida a velocidades específicas do vento, força de elevação para diferentes velocidades e, em geral, sua aerodinâmica. Em Dayton, Ohio \cite{vertical1946}, foi construído um túnel de vento vertical que possibilita que sejam feitos testes específicos para aeronaves. Um objeto é posto em queda livre virtual devido a velocidade do vento impulsionado por uma turbina e se observa o seu movimento na situação. Este teste foi essencial para a prototipagem de aviões militares e comerciais desde os anos 50.


Hoje, os testes ainda são responsáveis por economizar milhões de reais na fase de prototipagem em testes aerodinâmicos usados no desenvolvimento de trens de alta velocidade \cite{kwon2001}. Em objetos que se movem em alta velocidade, o vento se torna um fator preocupante. Uma rajada de vento pode ser o fator determinante em um descarrilhamento de um trem. Os testes possibilitam que os cientistas analisem o  comportamento de suas criações em situações previsíveis e os ajuda a antecipar possíveis problemas.

No entanto, existe um problema no teste aerodinâmico utilizando túneis de vento. Para os testes de queda livre como o de aeronaves, o controle de altitude do objeto é complexo. É possível obter o número de Reynolds, uma variável sem dimensão que possibilita predizer o padrão de fluxo de fluídos, e é possível medir a velocidade do vento atuando em cima do objeto. Porém, a real dificuldade se encontra no fato de que quando a força do vento atua em cima de um objeto, ele tende a girar \cite{briggs1959} de acordo com suas características aerodinâmicas, o que não é possível de se controlar sem um atuador no objeto.


Tendo em vista que o controle de altura de um objeto tendo como único atuador o vento, e que um modelo de atuação da força do vento sobre um objeto é complexo e de baixa fidelidade, e que precisamos de um modelo que possibilite um controle de altitude preciso e robusto, temos como opção a modelagem caixa-preta, conhecida como identificação de sistemas.


\section{Objetivos}

\subsection{Objetivo Geral}

O objetivo deste trabalho é construir uma plataforma de estudos de túnel de ar, aplicar identificação por subespaços e por mínimos quadrados, e projetar um controlador para o sistema.

\subsection{Objetivos Específicos}

Para alcançar o objetivo geral deste projeto, devemos primeiro alcançar os seguintes objetivos específicos.

\begin{itemize}
	\item Construir uma plataforma de experimento no formato de túnel de vento;
	\item Medir a velocidade do vento em alturas variadas dentro do túnel de vento;
	\item Criar um simulador para o túnel de vento;
	\item Executar testes para identificação usando como sinal de controle um sinal binário pseudo aleatório (PRBS);
	\item Identificar o sistema por mínimos quadrados;
	\item Identificar o sistema por subespaços;
	\item Validar os modelos obtidos;
	\item Obter um controlador para os dois modelos usando alocação de polos.
\end{itemize}

\section{Organização do trabalho}


Este trabalho se encontra organizado em seis capítulos com várias subseções. 
\begin{itemize}
	\item Capítulo 2: é a revisão bibliográfica.A fundamentação teórica dividida em conceitos fundamentais ao entendimento do trabalho, explicações sobre modelos de sistemas dinâmicos, dos critérios de estabilidade de sistemas discretos, e da identificação de sistemas e estimação de parâmetros.
	\item Capítulo 3: descreve a modelagem e simulação do túnel de vento, onde se explica o processo de modelagem utilizado e a construção de um simulador no Simulink.
	\item Capítulo 4: trata da identificação do túnel usando mínimos quadrados e subespaços, explicando os procedimentos usados e o método de validação do modelo.
	\item Capítulo 5: explica como foi projetado o controlador para adequar o sistema aos requisitos de sua resposta ao degrau unitário.
	\item Capítulo 6: trata da avaliação experimental do sistema de controle, onde o sistema é testado para avaliar sua robustez.
	\item Capítulo 7: é a conclusão do trabalho onde as considerações finais e trabalhos futuros serão discutidos.
\end{itemize}     

% Fim Capítulo