\chapter*{Resumo}
\thispagestyle{empty}

A identificação caixa-preta é uma forma de identificar um modelo matemático para um sistema que, em alguns casos, não pode ser modelado de outra forma. O controle de altura de um objeto dentro de um túnel de vento é um desses casos onde a modelagem não é suficiente. Este trabalho estuda a identificação do túnel de vento por mínimos quadrados e por subespaços para aplicar um controle por realimentação de estados.

\vspace{50pt}

\paragraph{Palavras-chave:} Levitador, Túnel de Vento, Controle de Sistemas.
